\subsection*{Introducing \hyperlink{a00040}{Yarn} Dialogue}

\hyperlink{a00040}{Yarn} Dialogue is designed for Authors who have little or no programming knowledge. It makes no assumptions about how your game presents dialogue to the player, or about how the player chooses their responses. These tasks are left to the game production team, using programming and design tools such as Unity3\-D and C\#.

\subsubsection*{Simple Example Dialogue}

In \hyperlink{a00118}{our simple example}, we will create a very simple Dialogue that demonstrates the very basic usage of \hyperlink{a00040}{Yarn}.

\subsubsection*{Complex Example Dialogue}

In the \hyperlink{a00117}{complex example}, we will create a slightly more complex Dialogue to the simple example above. This example demonstrates more advanced possibilities that exist within telling stories in a narrative game environment.

\subsubsection*{Localisation}

Our localisation document describes how localisation, or translating the text into other languages, takes place. \begin{quotation}
$\ast$$\ast$$\ast$\-Note\-:$\ast$$\ast$$\ast$ At the moment, it is a requirement that a programmer run a special tool across the Dialogue to produce the files to be translated.

\end{quotation}


\subsubsection*{General Usage}

We have a page of \hyperlink{a00116}{general usage}, describing many further features of the \hyperlink{a00040}{Yarn} Language. It also containins small Dialogue snippets as well as tips and suggestions you might find useful. 